\documentclass[a4paper,12pt]{article}
\usepackage{CJKutf8}
\usepackage{amsmath}
\usepackage{graphicx}
\usepackage{float}
\usepackage{array}
\usepackage{geometry}
\usepackage{caption}
\geometry{margin=1in}

\begin{document}
\begin{CJK}{UTF8}{gbsn}
\title{计算机视觉实验2-1:几何变换与图像金字塔}
\author{CVPR 实验}
\date{\today}
\maketitle

\section{实验目的}
\begin{itemize}
    \item 掌握图像的参数化几何变换原理;
    \item 掌握图像的前向变换与逆向变换;
    \item 理解图像的下抽样与插值方法;
    \item 掌握高斯金字塔与拉普拉斯金字塔的构造原理。
\end{itemize}

\section{实验原理简述}
\subsection{二维几何变换}
齐次坐标下的二维几何变换可统一写为 $\mathbf{p}' = T\mathbf{p}$。常见变换矩阵如表\ref{tab:transform} 所示。逆变换 $T^{-1}$ 将目标坐标映射回源坐标,便于插值采样。

\begin{table}[H]
    \centering
    \caption{常见二维几何变换矩阵}
    \label{tab:transform}
    \begin{tabular}{|m{3cm}|m{7cm}|m{2cm}|}
        \hline
        变换 & 矩阵 $T$ & 自由度 \\
        \hline
        平移 & $\begin{bmatrix}1&0&t_x\\0&1&t_y\\0&0&1\end{bmatrix}$ & 2 \\
        \hline
        旋转 & $\begin{bmatrix}\cos\theta&-\sin\theta&0\\ \sin\theta&\cos\theta&0\\0&0&1\end{bmatrix}$ & 1 \\
        \hline
        欧氏 & 在旋转基础上叠加平移 & 3 \\
        \hline
        相似 & 旋转+等比缩放+平移 & 4 \\
        \hline
        仿射 & $\begin{bmatrix}a&b&c\\d&e&f\\0&0&1\end{bmatrix}$ & 6 \\
        \hline
    \end{tabular}
\end{table}

\subsection{前向与逆向映射}
前向映射将源像素投射到新坐标,容易出现空洞;逆向映射逐像素查询源坐标,再通过插值得到灰度,可避免空洞。

\subsection{插值方式}
近邻插值选择最近的整数网格点;双线性插值对周围四个网格点按距离进行加权平均,过渡更平滑。

\subsection{高斯与拉普拉斯金字塔}
高斯金字塔通过高斯滤波后下采样逐层构造;拉普拉斯金字塔为相邻高斯层的差值(将低分辨率层上采样后相减),记录细节残差,便于重建。

\section{实验环境与程序设计}
本实验使用纯 Python 实现,无外部依赖。图像以三通道列表形式存储,自行实现 PNG 写入器、几何变换、卷积、高斯/拉普拉斯金字塔等函数。代码位于 \texttt{2-1/experiment.py}。\textbf{为减小仓库体积,仓库未附带生成的 PNG 图像,编译报告前请先运行 \texttt{python experiment.py} 以生成 \texttt{output/} 目录中的图片。}

\section{实验步骤与结果}
\subsection{测试图像}
生成一幅 $256\times256$ 的合成图像,包含矩形、圆形、三角形和网格线,便于观察几何变化。如图\ref{fig:original}。

\begin{figure}[H]
    \centering
    \includegraphics[width=0.42\textwidth]{output/original.png}
    \caption{合成测试图像}
    \label{fig:original}
\end{figure}

\subsection{前向与逆向映射对比}
对平移、旋转、欧氏、相似和仿射变换分别进行前向映射与逆向映射。前向映射仅做整数落点,容易出现黑色空洞;逆向映射结合插值后图像连贯性更好。各变换矩阵的运行结果如图\ref{fig:translation}--\ref{fig:affine} 所示,均展示了前向映射、逆向近邻和逆向双线性三种输出。

\begin{figure}[H]
    \centering
    \includegraphics[width=0.32\textwidth]{output/forward_translation.png}
    \includegraphics[width=0.32\textwidth]{output/inverse_nearest_translation.png}
    \includegraphics[width=0.32\textwidth]{output/inverse_bilinear_translation.png}
    \caption{平移变换:前向映射(左)、逆向近邻(中)、逆向双线性(右)}
    \label{fig:translation}
\end{figure}

\begin{figure}[H]
    \centering
    \includegraphics[width=0.32\textwidth]{output/forward_rotation.png}
    \includegraphics[width=0.32\textwidth]{output/inverse_nearest_rotation.png}
    \includegraphics[width=0.32\textwidth]{output/inverse_bilinear_rotation.png}
    \caption{旋转变换:前向映射(左)、逆向近邻(中)、逆向双线性(右)}
    \label{fig:forward_inverse}
\end{figure}

\begin{figure}[H]
    \centering
    \includegraphics[width=0.32\textwidth]{output/forward_euclidean.png}
    \includegraphics[width=0.32\textwidth]{output/inverse_nearest_euclidean.png}
    \includegraphics[width=0.32\textwidth]{output/inverse_bilinear_euclidean.png}
    \caption{欧氏变换(旋转+平移):前向映射与逆向映射对比}
    \label{fig:euclidean}
\end{figure}

\begin{figure}[H]
    \centering
    \includegraphics[width=0.32\textwidth]{output/forward_similarity.png}
    \includegraphics[width=0.32\textwidth]{output/inverse_nearest_similarity.png}
    \includegraphics[width=0.32\textwidth]{output/inverse_bilinear_similarity.png}
    \caption{相似变换:前向映射与逆向映射对比}
    \label{fig:similarity}
\end{figure}

\begin{figure}[H]
    \centering
    \includegraphics[width=0.32\textwidth]{output/forward_affine.png}
    \includegraphics[width=0.32\textwidth]{output/inverse_nearest_affine.png}
    \includegraphics[width=0.32\textwidth]{output/inverse_bilinear_affine.png}
    \caption{仿射变换:前向映射与两种插值的逆向映射}
    \label{fig:affine}
\end{figure}

\subsection{插值方式对比}
近邻插值会产生阶梯状边缘;双线性插值过渡平滑,网格和边缘锯齿明显减少。图\ref{fig:forward_inverse} 右图即可观察到差异。

\subsection{高斯与拉普拉斯金字塔}
采用 $5\times5$ 高斯核进行平滑,再按 2 倍下采样构建 4 层高斯金字塔。拉普拉斯金字塔通过上采样并相减得到,早期版本直接加 128 偏移显示,导致残差为正时全图偏白。本次对残差按通道对称归一化到 $[0,255]$(以 $\lVert\Delta\rVert_{\infty}$ 为尺度),同时再加 $0.5$ 偏移展示正负,避免了白蒙蒙的效果。部分结果如图\ref{fig:gaussian}、图\ref{fig:laplacian}。

\begin{figure}[H]
    \centering
    \includegraphics[width=0.3\textwidth]{output/gaussian_level_0.png}
    \includegraphics[width=0.3\textwidth]{output/gaussian_level_1.png}
    \includegraphics[width=0.3\textwidth]{output/gaussian_level_2.png}
    \caption{高斯金字塔第0、1、2层}
    \label{fig:gaussian}
\end{figure}

\begin{figure}[H]
    \centering
    \includegraphics[width=0.3\textwidth]{output/laplacian_level_0.png}
    \includegraphics[width=0.3\textwidth]{output/laplacian_level_1.png}
    \includegraphics[width=0.3\textwidth]{output/laplacian_level_2.png}
    \caption{拉普拉斯金字塔第0、1、2层(显示已偏移)}
    \label{fig:laplacian}
\end{figure}

\section{结果分析与讨论}
\begin{itemize}
    \item \textbf{前向 vs. 逆向:} 前向映射难以填补空洞,需要额外插值或扩散;逆向映射逐像素查询源坐标,能保证输出完整。
    \item \textbf{插值差异:} 近邻插值速度快但易出现锯齿,双线性插值能平滑网格与边缘,适合几何校正。
    \item \textbf{金字塔:} 高斯金字塔有效抑制高频后再下采样,拉普拉斯金字塔记录各层细节,可用于多尺度融合或重建;若直接加常数偏移展示残差,正值占优时会让图像整体偏白,因此需按残差极值对称归一化。
\end{itemize}

\section{结论}
实验实现了常见几何变换、前向/逆向映射与两种插值方式,并构建了高斯与拉普拉斯金字塔。结果验证了逆向映射配合双线性插值的优势,也展示了金字塔对多尺度表示的作用。

\end{CJK}
\end{document}
