\documentclass[12pt,a4paper]{article}
\usepackage{ctex}
\usepackage{amsmath, amssymb}
\usepackage{graphicx}
\usepackage{float}
\usepackage{hyperref}
\usepackage{enumitem}
\setlist[itemize]{leftmargin=16pt}
\setlist[enumerate]{leftmargin=16pt}

\title{计算机视觉实验 2-2:边缘与角点检测}
\author{实验者:xxx}
\date{\today}

\begin{document}
\maketitle

\section{实验内容}
\begin{enumerate}
  \item 基于高斯一阶微分计算图像梯度(幅值与方向),分析高斯方差对梯度的影响;
  \item 实现 Canny 边缘检测器,展示各阶段的中间结果,分析参数影响;
  \item 实现 Harris 角点检测器,分析窗口参数影响并讨论不变性与等变性。
\end{enumerate}

\section{实验环境与数据}
\begin{itemize}
  \item 语言:Python 3,无额外第三方图像库;
  \item 代码位置:仓库 \texttt{2-2} 目录;
  \item 运行方式:\texttt{python 1.py}、\texttt{python 2.py}、\texttt{python 3.py};
  \item 结果保存:脚本会在 \texttt{outputs/} 下生成 PNG,已通过 \texttt{.gitignore} 排除,需本地运行脚本再编译报告。
\end{itemize}

\section{相关理论}
\subsection{高斯一阶微分梯度}
二维高斯核 $G_{\sigma}(x,y)$ 与其一阶微分 $\partial G/\partial x,\partial G/\partial y$ 分别用于平滑与求导:
\[
\nabla f_G = \left[\dfrac{\partial G}{\partial x} * f,\; \dfrac{\partial G}{\partial y} * f\right],\quad
\|\nabla f_G\| = \sqrt{I_x^2 + I_y^2},\quad
\theta = \arctan\frac{I_y}{I_x}.
\]
大方差提升平滑程度,抑制高频噪声但也会弱化细节。

\subsection{Canny 边缘检测}
流程:\textcircled{1} 高斯滤波降噪;\textcircled{2} 一阶差分求梯度幅值与方向;\textcircled{3} 沿梯度方向做非极大值抑制;\textcircled{4} 双阈值滞后连接强弱边缘。好的边缘检测器应具备检测率高、定位精度高、响应单一等特性。

\subsection{Harris 角点检测}
利用二阶矩阵 $M$ 的行列式与迹构建响应:
\[
R = \det(M) - \alpha \cdot (\operatorname{trace} M)^2,
\]
$R$ 同时反映窗口在水平方向与垂直方向的强梯度。角点需满足 $R>T$ 且为局部极大值。窗口(或平滑)尺度控制了响应的空间支持,从而影响角点数量与定位精度。

\section{实验步骤与结果概述}
\subsection{任务一:高斯一阶微分梯度}
\begin{itemize}
  \item 选择 $\sigma\in\{0.5,1.0,2.0\}$,计算 $G_x, G_y$ 卷积后的 $I_x,I_y$,再求幅值与方向;
  \item 幅值图随 $\sigma$ 增大而更平滑,细节与噪声被抑制;方向图用色相编码,平滑后方向更连续。
\end{itemize}
生成文件:\texttt{outputs/task1/gx\_sigma*.png}、\texttt{mag\_sigma*.png}、\texttt{orient\_sigma*.png}。

\subsection{任务二:Canny 边缘检测}
\begin{itemize}
  \item 采用高斯平滑(默认 $\sigma=1.0$),Sobel 求梯度;
  \item 沿量化梯度方向做 NMS;
  \item 以高阈值比例 \texttt{high\_ratio=0.25}、低阈值比例 \texttt{low\_ratio=0.1} 进行滞后阈值;
  \item 输出包含:模糊图、梯度幅值、方向、NMS 结果以及最终边缘二值图。
\end{itemize}
参数影响:
\begin{itemize}
  \item $\sigma$ 增大可降低噪声但使边缘略宽、漏检弱边缘;
  \item 高阈值抬升会减少检测到的边缘,低阈值抬升会切断弱边缘的连接;
  \item 边缘质量依赖梯度方向量化与 NMS,过粗的量化会造成边缘锯齿。
\end{itemize}

\subsection{任务三:Harris 角点检测}
\begin{itemize}
  \item 使用高斯导数求梯度,窗口平滑方差 \texttt{sigma\_window=1.5},$\alpha=0.05$;
  \item 以响应最大值的 1\% 作为阈值并做 3$\times$3 NMS;
  \item 生成原图角点可视化、缩放 0.7 倍以及旋转 $30^\circ$ 后的角点可视化,用于观察尺度与旋转等变性;
  \item 窗口方差越大,角点更稀疏且定位更平滑;阈值越高,角点数量越少但鲁棒性更强。
\end{itemize}

\section{结论与思考}
\begin{itemize}
  \item 高斯方差在梯度计算、Canny 与 Harris 中都体现为噪声抑制与细节保持的权衡;
  \item Canny 通过 NMS 与滞后阈值实现单一响应与连通性,但阈值选择需要结合图像对比度;
  \item Harris 角点具有一定的平移、旋转等变性,尺度变化需要匹配的平滑尺度以维持响应;
  \item 所有结果可通过运行脚本重新生成,便于针对不同参数进行扩展实验。
\end{itemize}

\end{document}
